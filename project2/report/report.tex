\documentclass[10pt,draftclsnofoot,peerreview,letterpaper,onecolumn,]{IEEEtran}
\def\@IEEEstringphv{phv}
\def\IEEEsetsidemargin 0.75
\usepackage[margin=0.75in]{geometry}
\newcommand{\tab}{\hspace*{2em}}
\usepackage[margin=0.75in]{geometry}
\usepackage{amsmath}
\usepackage{hyperref}
\usepackage{graphicx}
\usepackage{textcomp}
\usepackage{verbatim}
\usepackage{color}

\hypersetup{
	pdfborder = {0 0 0}
}
\parindent = 0.0 in
\parskip = 0.1 in

\begin{document}

\begin{titlepage} % Suppresses displaying the page number on the title page and the subsequent page counts as page 1
	
	\raggedleft % Right align the title page
	
	\rule{1pt}{\textheight} % Vertical line
	\hspace{0.05\textwidth} % Whitespace between the vertical line and title page text
	\parbox[b]{0.75\textwidth}{ % Paragraph box for holding the title page text, adjust the width to move the title page left or right on the page
		
		{\Huge\bfseries Project 2 }\\[2\baselineskip] % Title
		{\large\textit{Fall 2017 CS544 Operating System}}\\[4\baselineskip] % Subtitle or further description
		{\Large Weijie Mo} % Author name, lower case for consistent small caps
		
		\vspace{0.5\textheight} % Whitespace between the title block and the publisher
		
		{\noindent October 30th 2017}\\[\baselineskip] % Publisher and logo
        {\noindent Abstract:
    This report is a introduction of what I've done in this project, including problem and solutions, and some ideas after finished this project.}\\[\baselineskip] % Publisher and logo
	}

\end{titlepage}

\section{My design}

In designing out SSTF (Shortes Seek time first) I/O scheduler, the most important part is to find out a way to guide the direction of the block device's arm, the current sector position of the block device arm, as well as a way to merge similar I/O requests into a sequential block access.I learned the kernel build process, and how each .c file enabled for building, compiled, and linked into the kernel. The design choice helped me a lot in this project, which is identifying that the noop scheduler has been setup already, and therefore I could compile the sstf scheduler separately.
\
\section{Work log}
\begin{enumerate}
\item \emph{Fri Oct 27 2017} Start to read some books related to I/O scheduler.
\item \emph{Sat Oct 28 2017} Continue to read the book, besides I also get more familiar with elevator algorithm.
\item \emph{Sun Oct 29 2017} Modify the code and begin to write the report in Latex format.
\item \emph{Mon Oct 30 2017} Completed the Latex file and almost finish the design, to ensure there is no problem in the assignment, test the code several times.
\end{enumerate}


\section{Version control log}

\begin{table}[h]
\centering
\begin{tabular}{|l|l|l|l|}
\hline

Author    & Date & Commit message
\\ \hline
Weijie Mo & Mon Oct 30 2017 & finish the sstf scheduler    557319430f262d6724d6b7a908d89f5f78c21311
\\ \hline

\end{tabular}
\end{table}

\section{Questions}
\subsection{What do you think the main point of this assignment is?}
To be acquainted with a I/O schedule algorithm, and cultivate the ability to read and design new algorithms.

\subsection{How did you personally approach the problem? Design decisions, algorithm, etc.}
In the first step, I spent a large amount of time reading the materials about I/O schedulers, which helped me know more about conceptual background to begin step two, which was analysing the source code of the existing schedulers. Once I felt somewhat confident with the material I began step three, which was implementation.

\subsection{How did you ensure your solution was correct? Testing details, for instance.}
I compiled SSTF scheduler, then I tested the file by qemu, the VM booted up without any issue, so I could be sure there is no big problem in this solution.

\subsection{What did you learn?}
I have learned lots of new knowledge in this assignment, and solidify what I have learned in last assignment. In last assignment I have learned how to install a VM, build my new kernel as required, I also acquainted many new tools like gdb, gcc, makefile, mobaXterm, cgwin, git, etc. Luckily, most of them I could solidify in this assignment, and strengthen the impression of these knowledge. This assignment, I also learned how to create a design used to implement the new algorithm.

\subsection{How should the TA evaluate your work? Provide detailed steps to prove correctness.}
When run the VM, you may see following words: KERN INFO SSTF add .... which means it was called.




\end{document} 