\documentclass[journal, letterpaper, draftclsnofoot, onecolumn, 10pt]{IEEEtran}

\usepackage{graphicx}
\usepackage{amssymb}
\usepackage{amsmath}
\usepackage{amsthm}

\usepackage{alltt}
\usepackage{float}
\usepackage{color}
\usepackage{url}
\usepackage{listings}

\usepackage{balance}
\usepackage[TABBOTCAP, tight]{subfigure}
\usepackage{enumitem}
\usepackage{pstricks, pst-node}
\usepackage{placeins}
\usepackage{geometry}
\usepackage{hyperref}
\geometry{textheight=8.5in, textwidth=6in}

\lstset{
  language=C,                % choose the language of the code
  numbers=left,                   % where to put the line-numbers
  stepnumber=1,                   % the step between two line-numbers.
  numbersep=5pt,                  % how far the line-numbers are from the code
  backgroundcolor=\color{white},  % choose the background color. You must add \usepackage{color}
  showspaces=false,               % show spaces adding particular underscores
  showstringspaces=false,         % underline spaces within strings
  showtabs=false,                 % show tabs within strings adding particular underscores
  tabsize=8,                      % sets default tabsize to 2 spaces
  captionpos=b,                   % sets the caption-position to bottom
  breaklines=true,                % sets automatic line breaking
  breakatwhitespace=true,         % sets if automatic breaks should only happen at whitespace
  title=\lstname,                 % show the filename of files included with \lstinputlisting;
}

%random comment

\newcommand{\cred}[1]{{\color{red}#1}}
\newcommand{\cblue}[1]{{\color{blue}#1}}

\newcommand{\toc}{\tableofcontents}


\def\name{Leon Leighton, Alec Merdler, Arthur Shing}

%% The following metadata will show up in the PDF properties
\hypersetup{
   colorlinks = true,
   urlcolor = black,
   linkcolor = black,
   pdfauthor = {\name},
   pdfkeywords = {cs444 ``operating systems'' slob slab best-fit},
   pdftitle = {CS 444 Project 4: The SLOB SLAB},
   pdfsubject = {CS 444 Project 4},
   pdfpagemode = UseNone
}

\parindent = 0.0 in
\parskip = 0.1 in


\begin{document}
\title{Project 4: The SLOB SLAB}
\author{Leon Leighton, Alec Merdler, and Arthur Shing}

\begin{titlepage}
    \pagenumbering{gobble}
    \centering
    \maketitle
    \begin{abstract}
      This document provides an overview of the work done by Group 11-09 for Project 4: The SLOB SLAB.
      It includes the design we used for implementing a best-fit algorithm into slob.c.
      It also includes answers to the questions included in the assignment and details on the testing procedure.
    \end{abstract}


\end{titlepage}
\pagenumbering{arabic}
\tableofcontents
\clearpage

\section{Project 4}

\subsection{Design}

% Talk about the design and algorithm here
The existing SLOB allocator uses a first-fit algorithm. What this means is that the first available page with enough memory is used for allocation.
To change this into a best-fit algorithm, we will change the algorithm so that it does not stop and allocate upon finding a suitable page.
Instead, the list will be iterated through in search of a suitable page, and each time a more suitable page is found it will choose that one instead.
% To test, we will


\subsection{Answers}

\subsubsection{What do you think the main point of this assignment is?}

We think that the main point of this assignment was to learn how the Linux kernel allocates memory, particularly with the SLOB allocator.
In this assignment, we modified the SLOB allocator algorithm from a first-fit to a best-fit algorithm, in hopes to optimize performance and avoid memory fragmentation. \\


\subsubsection{How did you personally approach the problem?}



To design a working implentation for the best-fit allocation algorithm, we first inspected the existing slob.c code to see how the first-fit algorithm worked.
We saw that the slob was iterating through the list of partially free pages, and skipping allocation attempts for pages that didn't have enough room until a suitable page was found.
To change this to a best-fit algorithm, we moved the allocation attempt to outside the list\_for\_each\_entry() iteration, and added a new variable sp\_best that holds the current most suitable page.
Upon iterating through the whole list of partially free pages, the most suitable page would be found and an allocation attempt could be made to it. If no suitable page was found, then a new page is allocated. \\


\subsubsection{Testing Details}

% TESTING DETAILS
To test our implementation, we included two syscalls that return total and used memory. These were defined respectively as mem\_size and mem\_used.
We created a test program (test.c) that allocates memory into buffers prints fragmentation and used/total memory. We used this to compare the fragmentation differences between the two algorithms.
The results can be found in the file hw4-stats, and are as follows: \\
\clearpage
\begin{verbatim}
    bestfit:
    psize: 227639296
    pused:  26737030
    11.75%

    firstfit:
    psize: 27717632
    pused: 10662801
    38.47%
\end{verbatim}

The variable psize is the total amount of space, and pused is the amount of used space. As it stands, we found that the best-fit algorithm had a lower fragmentation than the first-fit.\\

\subsubsection{What did you learn?}

Overall, we learned about the SLOB allocator in the Linux kernel and how the kernel allocates memory.
We also learned about first-fit and best-fit algorithms, and how a best-fit algorithm can avoid loss of performance. \\


\subsection{Work Log}
We looked into the existing slob.c code before we came together to design our implentation.
We came up with our design on the afternoon of June 8th and began work in the evening. We implemented our changes to the SLOB algorithm and worked on testing the next morning.


\clearpage
\subsection{Version Control Log}



\FloatBarrier



\scalebox{0.8}{
\begin{tabular}{l l l l}\textbf{Detail} & \textbf{Author} & \textbf{Description} & \textbf{Date}\\\hline
\href{https://github.com/alecmerdler/cs-444-group-119/commit/f55ce2e36853dfbdd52bed59cfa2ce43df645ed3}{f55ce2e} & Leon Leighton & Change way of gathering stats & Fri Jun 9 13:53:59 2017 -0700\\\hline
\href{https://github.com/alecmerdler/cs-444-group-119/commit/127e114c2a48cf4896cfb1b315297ece28c49e9a}{127e114} & Arthur Shing & Changes to doc and vclog script & Fri Jun 9 12:42:27 2017 -0700\\\hline
\href{https://github.com/alecmerdler/cs-444-group-119/commit/b926dbbf8686c2cd06d0a6a08ef449f8da4a1d3b}{b926dbb} & Leon Leighton & Made changes to syscall to get total number of pages & Fri Jun 9 12:15:24 2017 -0700\\\hline
\href{https://github.com/alecmerdler/cs-444-group-119/commit/69f5ed6b069d281e822cf1e43773cb847eb869ac}{69f5ed6} & Leon Leighton & Fixing comparison of units & Fri Jun 9 02:13:16 2017 -0700\\\hline
\href{https://github.com/alecmerdler/cs-444-group-119/commit/ee59ea6a380cd3042481a5bb41c9cfc9c06d01c7}{ee59ea6} & Leon Leighton & Updates to slob.c and beginning of syscall & Fri Jun 9 02:00:03 2017 -0700\\\hline
\href{https://github.com/alecmerdler/cs-444-group-119/commit/c5b5d76787fd467cb08813cdd5c84f8109bc9f5b}{c5b5d76} & Arthur Shing & Fixed typo & Fri Jun 9 01:35:09 2017 -0700\\\hline
\href{https://github.com/alecmerdler/cs-444-group-119/commit/7a02bf6ba44b86597443b726e837b34e8121e113}{7a02bf6} & Arthur Shing & Added answers to doc & Fri Jun 9 01:30:59 2017 -0700\\\hline
\href{https://github.com/alecmerdler/cs-444-group-119/commit/424093b9831ffbbcce981e57958bbab72e65e54e}{424093b} & Leon Leighton & Change slob.c to best-fit & Thu Jun 8 22:11:23 2017 -0700\\\hline
\href{https://github.com/alecmerdler/cs-444-group-119/commit/eea6eab255a3a63eacbd486315f8270b52d2e080}{eea6eab} & Arthur Shing & Added hw4 tex file & Thu Jun 8 16:53:41 2017 -0700\\\hline
\href{https://github.com/alecmerdler/cs-444-group-119/commit/7cbb6d989006f504db0136fea5da32d500147e53}{7cbb6d9} & Arthur Shing & working cigarette smokers & Thu Jun 8 04:15:56 2017 -0700\\\hline
\href{https://github.com/alecmerdler/cs-444-group-119/commit/fa5dc9df5a990f40c8a55aea751347c8a07ace96}{fa5dc9d} & alecmerdler & changed BoundedSemaphore to Semaphore & Tue Jun 6 11:30:24 2017 -0700\\\hline
\href{https://github.com/alecmerdler/cs-444-group-119/commit/09f081eba62c5193144d3c17234b4bfa527d36de}{09f081e} & alecmerdler & more pusher code & Tue Jun 6 11:19:24 2017 -0700\\\hline
\href{https://github.com/alecmerdler/cs-444-group-119/commit/2b2ddac696610404faf0a860808d26c4415ad19e}{2b2ddac} & alecmerdler & more pusher code & Tue Jun 6 11:18:49 2017 -0700\\\hline
\href{https://github.com/alecmerdler/cs-444-group-119/commit/23bef706b9aa65a8a6c150131ec8baa2c97ce181}{23bef70} & alecmerdler & more work & Mon Jun 5 23:35:52 2017 -0700\\\hline
\href{https://github.com/alecmerdler/cs-444-group-119/commit/4a8b66bf7c19f0edef2378adf32d3f473b3a6abf}{4a8b66b} & alecmerdler & starting concurrency5 & Mon Jun 5 23:28:56 2017 -0700\\\hline
\href{https://github.com/alecmerdler/cs-444-group-119/commit/69cf70e6130d3129f7754feedf61ba70ed5453f8}{69cf70e} & Lee Leighton & Add tarball for concurrency4 & Thu Jun 1 11:58:13 2017 -0700\\\hline
\href{https://github.com/alecmerdler/cs-444-group-119/commit/583ecbdff859096a4521cf0b86a25b3ef0121b2e}{583ecbd} & alecmerdler & I'll do it for you pattern & Thu Jun 1 11:48:44 2017 -0700\\\hline
\href{https://github.com/alecmerdler/cs-444-group-119/commit/f25f71d59d6fe961e77c659c54655b57acb4765e}{f25f71d} & Arthur Shing & Added barbershop implementation & Thu Jun 1 05:02:49 2017 -0700\\\hline
\href{https://github.com/alecmerdler/cs-444-group-119/commit/c0438b035bac02f4a0b2bc5c5789690eabb88d0d}{c0438b0} & Arthur Shing & Not sure how to approach & Tue May 30 12:38:53 2017 -0700\\\hline
\href{https://github.com/alecmerdler/cs-444-group-119/commit/390e7a9b7c0fd15d4c15d2d6042b00ac2550266d}{390e7a9} & alecmerdler & merged with master & Tue May 30 12:30:48 2017 -0700\\\hline
\href{https://github.com/alecmerdler/cs-444-group-119/commit/ec16f6bf04069d333fec28f2014be9c952733564}{ec16f6b} & alecmerdler & starting concurrency 4 & Tue May 30 12:30:29 2017 -0700\\\hline\end{tabular}
}


\FloatBarrier
\end{document}
