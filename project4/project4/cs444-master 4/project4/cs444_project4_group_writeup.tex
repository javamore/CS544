%%%%%%%%%%%%%%%%%%%%%%%%%%%%%%%%%%%%%%%%%
% Short Sectioned Assignment
% LaTeX Template
% Version 1.0 (5/5/12)
%
% This template has been downloaded from:
% http://www.LaTeXTemplates.com
%
% Original author:
% Frits Wenneker (http://www.howtotex.com)
%
% License:
% CC BY-NC-SA 3.0 (http://creativecommons.org/licenses/by-nc-sa/3.0/)
%
%%%%%%%%%%%%%%%%%%%%%%%%%%%%%%%%%%%%%%%%%

%----------------------------------------------------------------------------------------
%	PACKAGES AND OTHER DOCUMENT CONFIGURATIONS
%----------------------------------------------------------------------------------------

\documentclass[paper=a4, fontsize=11pt]{scrartcl} % A4 paper and 11pt font size

\usepackage[T1]{fontenc} % Use 8-bit encoding that has 256 glyphs
\usepackage[english]{babel} % English language/hyphenation
\usepackage{amsmath,amsfonts,amsthm} % Math packages
\usepackage{graphicx}


\usepackage{courier}
\usepackage{listings}
\lstset{
         basicstyle=\footnotesize\ttfamily, 
         numberstyle=\tiny,          
         numbersep=5pt,             
         tabsize=2,                
         extendedchars=true,      
         breaklines=true,        
         showspaces=false,      
         showtabs=false,       
         xleftmargin=17pt,
         framexleftmargin=17pt,
         framexrightmargin=5pt,
         framexbottommargin=4pt,
         showstringspaces=false 
 }
 \lstloadlanguages{
         C
 }


\usepackage{lipsum} % Used for inserting dummy 'Lorem ipsum' text into the template

\usepackage{sectsty} % Allows customizing section commands
\allsectionsfont{\centering \normalfont\scshape} % Make all sections centered, the default font and small caps

\usepackage{fancyhdr} % Custom headers and footers
\pagestyle{fancyplain} % Makes all pages in the document conform to the custom headers and footers
\fancyhead{} % No page header - if you want one, create it in the same way as the footers below	
\fancyfoot[L]{} % Empty left footer
\fancyfoot[C]{} % Empty center footer
\fancyfoot[R]{\thepage} % Page numbering for right footer
\renewcommand{\headrulewidth}{0pt} % Remove header underlines
\renewcommand{\footrulewidth}{0pt} % Remove footer underlines
\setlength{\headheight}{13.6pt} % Customize the height of the header

\numberwithin{equation}{section} % Number equations within sections (i.e. 1.1, 1.2, 2.1, 2.2 instead of 1, 2, 3, 4)
\numberwithin{figure}{section} % Number figures within sections (i.e. 1.1, 1.2, 2.1, 2.2 instead of 1, 2, 3, 4)
\numberwithin{table}{section} % Number tables within sections (i.e. 1.1, 1.2, 2.1, 2.2 instead of 1, 2, 3, 4)

\setlength\parindent{0pt} % Removes all indentation from paragraphs - comment this line for an assignment with lots of text

%----------------------------------------------------------------------------------------
%	TITLE SECTION
%----------------------------------------------------------------------------------------

\newcommand{\horrule}[1]{\rule{\linewidth}{#1}} % Create horizontal rule command with 1 argument of height

\title{	
\normalfont \normalsize 
\textsc{Oregon State University College of Engineering} \\ [25pt] % Your university, school and/or department name(s)
\horrule{0.5pt} \\[0.4cm] % Thin top horizontal rule
\huge Assignment 3 \\ % The assignment title
\horrule{2pt} \\[0.5cm] % Thick bottom horizontal rule
}

\author{Zachary Murphy, Michael Woffendin, Patrick Hulligan} % Your name

\date{\normalsize\today} % Today's date or a custom date

\begin{document}

\maketitle % Print the title

%----------------------------------------------------------------------------------------
%	PROBLEM 1
%----------------------------------------------------------------------------------------

\section*{Design}

First, we clearly defined each of the two algorithms. First-fit finds the first available space that
 is the same size or larger than the information being written. Best-fit is the same, except it 
searches all pages and tries to be as close as possible to the target size. 

Next, we made a plan for comparing the two algorithms. We decided that the simplest way to put 
these algorithms to work was by allocating / freeing integer arrays of various sizes. To ensure 
that the sizes themselves did not skew the fragmentation results, both algorithms had to use the 
same data sets. 

Last we had to decide how to implement the system call. Since most low-numbers are already taken, 
we chose a random unoccupied high number ()

\section* {Version Control Log}
\begin{tabular}{ c | c | l }

\hline
1 & 5/31 & Initial Commit \\ \hline 

2 & 5/31 & Fix bugs in fitting algorithms \\  \hline
 
3 & 5/31 & Begin work on comparison tool (unfinished) \\ \hline

4 & 6/1 & Fix crashes in comparison tool (unfinished) \\ \hline

4 & 6/1 & Finish comparison tool \\ \hline

4 & 6/1 & Fix style violations \\ \hline


\end{tabular}


\section*{Work Log}

\hline
We did our planning for the project on Saturday May 30th. We began our implementation on 
Sunday May 31st and finished work on Monday June 1st.


%------------------------------------------------


%------------------------------------------------

%----------------------------------------------------------------------------------------

%----------------------------------------------------------------------------------------

\end{document}
